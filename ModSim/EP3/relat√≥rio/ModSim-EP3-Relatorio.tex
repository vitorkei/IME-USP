\documentclass{article}
\usepackage[utf8]{inputenc}
\usepackage{titlesec}
\usepackage{graphicx}
\usepackage{caption}

\title{MAC0209 - Modelagem e Simulação \\
Relatório Exercício Programa 3}
\author{Henrique Cerquinho - 9793700 \\
João Pedro Miguel de Moura - 7971622 \\
Tomás Bezerra Marcondes Paim - 7157602 \\
Vítor Kei Taira Tamada - 8516250}
\date{03/07/2017}

\begin{document}

\maketitle

\section{Introdução}
\paragraph{}Este relatório diz respeito às simulações realizadas pelo grupo para analisar um sistema complexo, o autômato celular de modelo de trânsito. Os principais desafios foram conseguir analisar as simulações para que as corretas conclusões e inferências fossem feitas, uma vez que o código-fonte do simulador foi fornecido pelo livro.

\section{Método}
\paragraph{}Como exigido pelo enunciado do EP3, modificamos os programas Freeway e FreewayApp do \textit{open source physics} e respondemos os ítens (a) até (h) do exercício 14.5 de simulação de trânsito. Diferente dos EPs 1 e 2, o programa utilizado no EP3 foi em Java uma vez que o próprio livro utiliza bibliotecas e códigos auxiliares mais complexos que seriam de difícil adaptação para Python.

\paragraph{}Uma vez que o código precisava ser alterado de um item para o outro no exercício, por vezes radicalmente, vários códigos diferentes foram salvos em pastas diferentes para melhor organização e garantia de que os ítens seriam respondidos corretamente em seus respectivos contextos.

\paragraph{}Além disso, para gerar o diagrama fundamental do item 14.5a, foram utilizadas duas bibliotecas auxiliares, \texttt{StdDraw.java} e \texttt{StdIn.java}, do livro \textit{Introduction to Programming in Java} (http://introcs.cs.princeton.edu/java/stdlib/). Tais bibliotecas foram enviadas junto dos códigos-fonte e relatório. O diagrama é gerado pelo programa \texttt{fundamentalDiagram.java}, que utiliza o arquivo \texttt{input.txt} como entrada.

\paragraph{}No item (g) é pedido que se implemente uma via de duas faixas de mesmo sentido na simulação, juntamente com uma regra para troca de faixas dos veículos simulados. No nosso caso, o veículo troca de faixa se houver espaço livre na outra faixa e se houver algum carro à frente que estiver truncando seu movimento.

\paragraph{}A resposta dos ítens do exercício 14.5 estão no arquivo "Resposta dos exercícios.pdf".

\paragraph{}O código que responde aos ítens 14.5a - 14.5e está na pasta abcde e o código que responde aos ítens 14.5f - 14.5h está na pasta fgh.

\paragraph{}Todos os algoritmos foram implementados em Java.

\section{Verificação do Programa}
\paragraph{}O programa FreewayApp foi construído para ser rodado sem nenhum argumento, mas utiliza uma interface gráfica para alterar os parâmetros da simulação, bem como botão para iniciar/pausar, parar, avançar passo-a-passo e reiniciar os parâmetros para o padrão. A simulação em si aparece em duas janelas: uma mostrando passo-a-passo e outra mostrando cada passo até o momento enquanto couber na janela.

\section{Dados}
\paragraph{}Os dados arbitrários (ou seja, os que não são pedidos no enunciado do exercício) utilizados para realizar as simulações estão no arquivo Parâmetros.ods (um arquivo idêntico mas em formato .csv está incluído) enviado junto dos códigos-fonte e do relatório.

\section{Análise}
\paragraph{}Uma vez que há muitos valores que cada parâmetro de entrada poderia assumir, foi difícil e trabalhoso realizar testes diferentes o suficiente para que as conclusões não fossem excessivamente enviesadas. Além disso, não sabíamos bem que valores poderíamos chamar de "baixo", "médio" e "alto" para cada parâmetro e algumas inferências são muito dependentes da opinião de quem está analisando a simulação.

\section{Interpretação}
\paragraph{}Pudemos traçar uma ligação entre as regras do sistema e sua evolução. No caso mais simples dos itens 14.5(a) até o (e), que utilizam apenas uma via, sem regras adicionais, o sistema evoluí dependendo dos parâmetros iniciais, quanto maior for a proporção do tamanho da pista para com o número de veículos mais rapidamente o sistema evoluí até apresentar engarrafamentos. No caso dos exercícios (f) e (h), a regra adicional gera mais carros lentos e isso, consequentemente, aumenta a taxa de engarrafamentos do sistema. No exercício (g), no entanto, a regra adicional muda completamente a dinâmica do sistema, fazendo agora com que as duas pistas interajam entre si dependendo da regra de mudança de pista que foi implementada. A aparição de engarrafamentos neste caso é bem mais rara.
 

\section{Crítica}
\paragraph{}Definir um tamanho de pista grande o suficiente para que seu tamanho seja irrelevante para a simulação não é simples para o método de simulação em questão. Além disso, as regras para entrada e saída de rampas e mudança de pista são arbitrárias, sendo abstrações muito simples de situações reais. Desta maneira, um maior refinamento destas regras é necessário para melhorar a qualidade da simulação. 

\section{Log}
\textbf{(Sexta 30/06/2017)} Início do trabalho: leitura do enunciado, capítulo 14 e análise do código-fonte fornecido \\
\textbf{(Sábado 01/07/2017)} Transcrição do código-fonte fornecido. Primeira simulações realizadas. \\
\textbf{(Domingo  02/07/2017)} Realização das últimas simulações. \\
\textbf{(Segunda  03/07/2017)} Início e finalizção do relatório. Compilação dos códigos-fonte e dos dados de cada simulação, bem como inferências e conclusões. Entrega do EP3. \\

\section{Análise Crítica}
\paragraph{}O estudo do código-fonte fornecido, bem como a modificação do mesmo para atender aos itens do questionário, foram as atividades mais custosas do EP3. A simulação e execução do FreewayApp e a escrita do relatório e das respostas do questionário ocorreram sem dificuldades.


\section{Atribuições}
\textbf{Vídeo:} Tomás, Henrique \\
\textbf{Relatório:} João, Vítor \\
\textbf{Programação:} João, Vítor

\section{Vídeo dos experimentos}

https://www.youtube.com/watch?v=83lMofmKylw

\end{document}