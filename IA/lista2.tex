\documentclass{article}

\usepackage[brazilian]{babel}
\usepackage[utf8]{inputenc}
\usepackage[T1]{fontenc}
\usepackage{amsmath}
\usepackage{MnSymbol}
\usepackage{wasysym}
\usepackage{mdframed}
\usepackage[a4paper, total={6in, 10.5in}]{geometry}
\usepackage{graphicx}

\title{MAC0210 - Laboratório de Métodos Numéricos}
\author{Exercício-programa 1}
\date{ }


\begin{document}


13.15 a) Como a probabilidade de o teste estar incorreto é de $1%$ (ou seja, de o exame dar positivo dado que eu não tenho a doença é de 0.01) e a ocorrência da doença é de 1 em cada 10.000 pessoas da minha idade, ou seja, de $0.0001%$, então é mais provável que o exame tenha dado falso positivo do que eu ter de fato a doença.


\end{document}