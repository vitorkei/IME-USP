\documentclass{article}

\usepackage[brazilian]{babel}
\usepackage[utf8]{inputenc}
\usepackage[T1]{fontenc}
\usepackage{amsmath}
\usepackage{MnSymbol}
\usepackage{wasysym}
\usepackage{mdframed}
\usepackage[a4paper, total={6in, 10.5in}]{geometry}

\title{\textbf{MAC0413 - Tópicos avançados de programação orientada a objetos}}
\author{\textbf{Crítica PetroLimpa}}
\date{}

\begin{document}

\maketitle

\begin{center}
\textbf{\large{Trabalho de}}

Fábio Henrique Kiyoiti dos Santos Tanaka

\bigskip
\textbf{\large{Criticado por}}

Luiz Felipe Moumdjian Girotto (8941189)

Vítor Kei Taira Tamada (8516250)
\end{center}
\begin{flushleft}

\textbf{\LARGE{Pontos positivos}}

\begin{itemize}

\item Diagrama de classes deixa as classes bem claras, bem como seus respectivos atributos e métodos.

\item Segundo diagrama deixa o funcionamento do sistema bem claro.

\end{itemize}

\bigskip
\bigskip
\textbf{\LARGE{Dúvidas e incertezas}}

\begin{itemize}

\item 

\end{itemize}

\bigskip
\bigskip
\textbf{\LARGE{Pontos negativos}}

\begin{itemize}

\item Ao nosso ver, a classe Gasto total é desnecessária.

\item As caixas de classes não utilizam o padrão UML para nome da classe ("\texttt{Analisador de fraude}" ao invés de "\texttt{AnalisadorFraude}" por exemplo) e para atributos ("\texttt{Tipo da compra: string}" ao invés de "\texttt{+ tipoDaCompra : string}" por exemplo).

\end{itemize}

\bigskip
\bigskip
\textbf{\LARGE{Sugestões de melhorias}}

\begin{itemize}

\item A classe Gasto total não precisa do atributo \texttt{Unidade} se assumir que será usado apenas um tipo de moeda.

\item No segundo diagrama, colocar setas para indicar o sentido de leitura para facilitar o entendimento.

\end{itemize}
\end{flushleft}

\end{document}
