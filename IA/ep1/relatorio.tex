\documentclass{article}

\usepackage[brazilian]{babel}
\usepackage[utf8]{inputenc}
\usepackage[T1]{fontenc}
\usepackage{amsmath}
\usepackage{MnSymbol}
\usepackage{wasysym}
\usepackage{mdframed}
\usepackage[a4paper, total={6in, 10.5in}]{geometry}

\title{\textbf{MAC425/5739 - Inteligência Artificial}}
\author{\textbf{Vítor Kei Taira Tamada - 8516250}}
\date{\textbf{Exercício-programa 1 - Relatório}}

\begin{document}

\maketitle

\quad {\large \textbf{a) Compilar em tabelas as estatísticas das buscas implementadas}}

\begin{center}
	\begin{tabular}{c | l | c | c}
		\hline
			&
			\textbf{Size}&
			\textbf{Nós}&
			\textbf{Custo} \\ \hline
			& Tiny & 15 & 10 \\
			\textbf{DFS} & Medium & 146 & 130\\
			& Big & 390 & 210 \\ \hline
			
			& Tiny & 15 & 8 \\ 
			\textbf{BFS}& Medium & 269 & 68 \\
			& Big & 620 & 210 \\ \hline
			
			& Tiny & 64 & 8 \\
			\textbf{IDS - 1 \footnotemark } & Medium & 8699 & 68 \\
			& Big & 60211 & 210 \\ \hline
			
			& Tiny & 11 & 10 \\
			\textbf{IDS - 10 \footnotemark[\value{footnote}] } & Medium & 982 & 68 \\
			& Big & 6069 & 210 \\ \hline
			
			& Tiny & 15 & 8 \\
			\textbf{UCS} & Medium & 269 & 68 \\
			& Big & 620 & 210 \\ \hline

			& Tiny & 14 & 8 \\
			\textbf{A*} & Medium & 221 & 68 \\
			& Big & 549 & 210 \\ \hline
	\end{tabular}
	
	\bigskip
	\emph{Tabela 1: Estatísticas das buscas implementadas em termos de nós expandidos e custo em cada labirinto}
	
\end{center}

	\footnotetext{Nos testes de busca de aprofundamento iterativo, o limite de cada iteração aumentou em 1 na IDS - 1 e em 10 na IDS - 10. O intuito foi verificar a influência do aumento do limite da busca a cada iteração no número de nós expandidos}

\begin{flushright}
$\blacksquare$
\end{flushright}

% *************************************************************************************

\bigskip
\quad {\large \textbf{b) Discutir os méritos e desvantagens de cada método, no contexto dos dados obtidos}}

\bigskip
\qquad \textbf{I) Busca em profundidade (DFS):} A busca em profundidade tem a sorte como fator importante: dependendo de qual caminho ela decidir percorrer primeiro, pode obter uma solução (não necessariamente a melhor) em pouco ou muito tempo. No caso dos testes, ela retornou uma solução rapidamente (poucos nós expandidos), mas não as melhores (custo mais alto nos labirintos pequeno e médio). A vantagem é que os únicos nós armazenados na memória são os do caminho encontrado.

\bigskip
\qquad \textbf{II) Busca em largura (BFS):} A busca em largura retorna o menor caminho entre a origem e o destino caso ele exista. Para isso, ela verifica todos os nós a uma distância $i$ da origem, sendo $i = 1, 2, ...$ O maior problema deste método de busca é o uso excessivo da memória e a expansão de muitos nós que podem até mesmo estar se distanciando da solução.

\bigskip
\qquad \textbf{III) Busca de aprofundamento iterativo (IDS):} A busca de aprofundamento iterativo combina a economia de memória da DFS (os únicos nós armazenados são os do caminho encontrado entre a origem e o destino) com a completudo e menor caminho da BFS. O maior problema deste método de busca é o fato de ele verificar os nós mais próximos da origem diversas vezes, havendo certo gasto de tempo, como é evidente pelo número muito maior de nós expandidos se comparar com os resultados das outras buscas.

\bigskip
\qquad \textbf{IV) Busca de custo uniforme (UCS):} A busca de custo uniforme é idêntica a uma busca em largura se todas as ações tiverem o mesmo custo. Por outro lado, se os custos das ações variarem entre si, a UCS garante encontrar o caminho mais barato, não sendo esse necessariamente o mais curto. A desvantagem deste método é o de haver a possibilidade de ele agir da mesma forma que uma BFS, evidenciado pelos números desta busca terem sido idênticos aos da BFS, já que os labirintos testados tem o mesmo custo para qualquer ação.

\bigskip
\qquad \textbf{V) Busca heurística (A*):} A busca A* procura o menor caminho (se as ações tiverem o mesmo custo) ou o mais barato, expandindo o menor número de nós possível.

Comparando os resultados das buscas, o algoritmo A* é claramente o melhor (considerando que a busca em profundidade encontrou um caminho correto rapidamente por acaso): ele conseguiu encontrar caminho de menor custo com menos nós expandidos.

Para que esta busca seja, de fato, a melhor, é necessário desenvolver uma heurística para que o problema seja resolvido pelo caminho ótimo. Logo, diferente das buscas mencionadas anteriormente, ela não é uma fórmula pronta e, portanto, é mais complicada de se implementar corretamente.

\begin{flushright}
$\blacksquare$
\end{flushright}

% *************************************************************************************

\bigskip
\quad {\large \textbf{c) Responder as questões teóricas}}

\bigskip
\qquad \textbf{Questão 1 - Busca em profundidade:} Apesar de haver diversos nós expandidos no labirinto, o Pac-Man percorre apenas aqueles necessários para alcançar o objetivo. Como eu não tenho certeza de como o método \texttt{getSuccessors()} itera sobre os estados sucessores, posso apenas afirmar que o resultado em \texttt{tinyMaze} foi conforme o esperado enquanto no \texttt{mediumMaze} e no \texttt{bigMaze} os resultados foram melhores que o esperado.

\bigskip
\qquad \textbf{Questão 2 - Busca em largura:} A busca em largura encontra um caminho de custo mínimo contanto que todas as ações tenham o mesmo custo, uma vez que o caminho de menor custo é o mais curto nessas circunstâncias. Se o custo das ações variar, a busca em largura não necessariamente encontrará o caminho de custo mínimo.

\bigskip
\qquad \textbf{Questão 3 - Busca de custo uniforme:} Uma vez que a busca de custo uniforme sempre executa a ação de menor custo em cada estado, o agente \texttt{StayEastSearchAgent} mantem-se do lado leste do mapa pois a ação de ir para a esquerda (oeste) é a mais cara e a de ir para a direita (leste) é a mais barata (seja $(x, y)$ a coordenada do estado resultado da ação a ser tomada, a função de custo é $0.5^{x}$, ou seja, ir para a esquerda - reduzir o valor de $x$ - tem custo maior) dentre as quatro (norte, sul, leste e oeste). Analogamente, \texttt{StayWestSearchAgent} tem a ação de ir para a direita (leste) como a mais cara e a de ir para a esquerda (oeste) como a mais barata (neste caso, a função de custo é $2^{x}$, ou seja, ir para a direita - aumentar o valor de $x$ - tem custo maior).

\bigskip
\qquad \textbf{Questão 4 - Busca de aprofundamento iterativo:} Sem a modificação, a busca de aprofundamento iterativo não é ótima pois pode acabar expandindo menos nós, uma vez que um nó $n$ pode ser de profundidade máxima por um caminho $a$ e profundidade não-máxima por um outro caminho $b$ explorado depois de $a$. Se isso ocorrer sem a modificação, não haverá a devida expansão que passe por $n$.

\bigskip
\qquad \textbf{Questão 5 - Busca heurística:} O algoritmo A* encontra uma solução mais rapidamente pois considera não apenas o custo de cada ação como também um conhecimento prévio (heurística) que ajuda a determinar qual ação é de fato melhor tomar e não apenas aparenta ser.

A razão para se implementar uma heurística consistente é garantir que um determinado estado seja alcançado pela primeira vez utilizando o caminho ótimo.

\bigskip
\qquad \textbf{Questão 6 - Agentes adversariais:} O agente reativo tem mais problemas para ganhar porque move-se pensando apenas em se manter longe dos fantasmas e próximo da comida. Como ele também não enxerga muito longe, é mais facilmente encurralado. O agente reativo jogaria melhor se tivesse um campo de visão e conhecimento maior sobre o mapa, e soubesse dar a devida prioridade entre ficar longe do fantasma e ir atrás da comida de acordo com as respectivas distâncias.

\begin{flushright}
$\blacksquare$
\end{flushright}

% *************************************************************************************

\bigskip
\quad {\large \textbf{d) Sugerir possíveis melhorias ao sistema e relatar dificuldades:} As maiores dificuldades enfrentadas para se resolver o exercício envolveram a extração e uso das informações necessárias para resolver o exercício, tais como receber as coordenadas atuais do Pac-Man e dos fantasmas, e as ações possíveis, e como manipular essas informações para programar o comportamento dos agentes. Minha principal sugestão é maior clareza em como manipular os dados disponíveis para construir as buscas e comportamentos uma vez que imagino que essa não deveria ser a parte desafiadora do problema, mas sim a parte lógica.

\end{document}