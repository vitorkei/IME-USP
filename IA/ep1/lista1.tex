\documentclass{article}

\usepackage[brazilian]{babel}
\usepackage[utf8]{inputenc}
\usepackage[T1]{fontenc}
\usepackage{amsmath}
\usepackage{MnSymbol}
\usepackage{wasysym}
\usepackage{mdframed}
\usepackage[a4paper, total={6in, 10.5in}]{geometry}

\title{\textbf{MAC425/5739 - Inteligência Artificial}}
\author{\textbf{Vítor Kei Taira Tamada - 8516250}}
\date{\textbf{Lista de exercícios 1 - Busca e lógica proposicional}}

\begin{document}

\maketitle

\textbf{Exercício 3.6:}

\quad a.

\qquad$\bullet$ Início: Todas as regiões do mapa estão pintadas com a mesma cor

\qquad$\bullet$ Meta: Nenhuma região faz fronteira com outra de mesma cor

\qquad$\bullet$ Ação: Dada uma região A, mudar a cor da região adjacente B tal que as regiões A e B tenham cores diferentes

\qquad$\bullet$ Custo: Número de ações

\bigskip
\quad b.

\qquad$\bullet$ Início: Macaco não tem nenhuma bananda

\qquad$\bullet$ Meta: Macaco pegou todas as bananas

\qquad$\bullet$ Ação: Empilhar os engradados embaixo de uma banana

\qquad$\bullet$ Custo: Número de ações

\bigskip
\quad c.

\qquad$\bullet$ Início: Nenhum registro de entrada alimentando o arquivo

\qquad$\bullet$ Meta: Registro de entrada inválido encontrado

\qquad$\bullet$ Ação: Alimentar o programa com um registro de entrada

\qquad$\bullet$ Custo: Número de ações

\bigskip
\quad d.

\qquad$\bullet$ Início: Três jarras vazias

\qquad$\bullet$ Meta: Pelo menos uma das jarras contém um litro

\qquad$\bullet$ Ação: Transferir parcial ou completamente o conteúdo de uma jarra para outra ou de uma jarra para o chão

\qquad$\bullet$ Custo: Número de ações

\begin{flushright}
$\blacksquare$
\end{flushright}

% ************************

\bigskip
\textbf{Exercício 3.11:}

\quad R) Nó de busca é uma estrutura auxiliar utilizada em buscas, possuindo um custo, uma ação e uma altura associados a ele.

\begin{flushright}
$\blacksquare$
\end{flushright}

% ************************

\bigskip
\textbf{Exercício 3.14:}

\bigskip
\quad a) Falso, pois a busca em profundidade pode, por acaso, encontrar o caminho de menor custo em sua primeira ramificação, enquanto a busca A*, dependendo da heurística, pode expandir os nós de um caminho que se mostra ruim após alguns passos.

\bigskip
\quad b) Falso. $h(n) = 0$ não é uma heurística admissível para o quebra cabeça de 8 peças pois esse valor faria com que uma busca informada agisse de maneira idêntica a busca de custo uniforme - uma busca cega. O valor de $h(n)$ deve ser zero apenas quando n é a meta do problema.

\bigskip
\quad c)

\bigskip
\quad d) Verdadeiro, pois a busca em largura independe do custo de cada ação.

\bigskip
\quad e) Verdadeiro, uma vez que

\begin{flushright}
$\blacksquare$
\end{flushright}

% ************************

\bigskip
\textbf{Exercício 1 - P1 (2015):}

\bigskip
\quad 1.

\qquad$\bullet$ Início: Agente encontra-se na célula 6

\qquad$\bullet$ Meta: Agente está na célula 14 ou na célula 18

\qquad$\bullet$ Ação: Mover-se para a célular de cima, de baixo, da esquerda ou da direita

\qquad$\bullet$ Custo: O custo de uma ação para a célula 18 é de 4; caso contrário, o custo é de 2

\qquad$\bullet$ Tamanho do espaço de estados: 

\begin{flushright}
$\blacksquare$
\end{flushright}

% ************************

\end{document}