\documentclass{article}

\usepackage[brazilian]{babel}
\usepackage[utf8]{inputenc}
\usepackage[T1]{fontenc}
\usepackage{amsmath}
\usepackage{MnSymbol}
\usepackage{wasysym}
\usepackage{mdframed}
\usepackage[a4paper, total={6in, 10.5in}]{geometry}

\title{\textbf{MAC0413 - Tópicos avançados de programação orientada a objetos}}
\author{\textbf{Crítica PetroLimpa}}
\date{}

\begin{document}

\maketitle

\begin{center}
\textbf{\large{Trabalho de}}

Caio Costa Salgado

Diego Cardozo Sandrim

\bigskip
\textbf{\large{Criticado por}}

Luiz Felipe Moumdjian Girotto (8941189)

Vítor Kei Taira Tamada (8516250)
\end{center}

\begin{flushleft}
\textbf{\LARGE{Pontos positivos}}

\begin{itemize}

\item O projeto apresenta clareza visual com relação à sequência de ações a serem tomadas nos diversos níveis do processo.

\item É apresentada uma solução interessante para o problema.

\end{itemize}

\bigskip
\bigskip
\textbf{\LARGE{Dúvidas e incertezas}}

\begin{itemize}

\item Por conta da forma que os diagramas foram construídos, não sabemos os atributos e métodos das classes, nem suas relações umas com as outras.

\item Cada escritório possui seu próprio site onde as suspeitas de fraude são publicadas. Logo, as suspeitas não são publicadas no website da empresa?

\end{itemize}

\bigskip
\bigskip
\textbf{\LARGE{Pontos negativos}}

\begin{itemize}

\item Nosso objetivo era "uma análise crítica do projeto \textit{orientado a objetos}". Porém, do que nos está disponível, é difícil discernir decisões de orientação a objetos do projeto uma vez que não existem especificações de classes ou informações do gênero.

\item A solução apresentada, por mais que seja interessante, nos parece desnecessariamente custosa e faz com que o produto final não seja unificado.

\item Tivemos a impressão de que, enquanto o primeiro PDF (petrolimpa.pdf) diz que cada escritório tem seu próprio site para publicar as suspeitas de fraude, o segundo (petrolimpa-sequence.pdf) diz que todos utilizam o mesmo, havendo contradição entre os diagramas.

\end{itemize}

\bigskip
\bigskip
\textbf{\LARGE{Sugestões de melhorias}}

\begin{itemize}

\item Ser mais consistente com a notação utilizada: no petrolimpa-sequence.pdf, os objetos utilizam a notação de objeto (\texttt{:Unidade} por exemplo) enquanto os métodos não (\texttt{registraCompra} ao invés de \texttt{registraCompra()} por exemplo).

\item Uso de diagrama de classes para melhor discernir decisões relacionadas à orientação a objetos.

\end{itemize}
\end{flushleft}

\end{document}
